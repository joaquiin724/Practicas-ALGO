\documentclass[11pt,openany]{book}
% #1-Asignatura
% #2-Curso
% #3-Nombre
% #4-Link
% #5-Foto

\newcommand{\portada}[5]{
    \begin{titlepage}
        \begin{center}
            \vspace*{0.5cm}
            
            % Titulo con #1 lo mas grande posible
            {\Huge \textbf{#1}}

            
            \vspace{0.5cm}
            \LARGE
            Curso #2 
            
            \vspace{1cm}
            
            \Huge{\textbf{Grupo Viterbi}}

            \vspace{1cm}
            \includegraphics[width=0.6\textwidth]{assets/Img/UGR-Logo.png}
            
            \vspace{0.5cm}

            \huge
            PRÁCTICA 1-ANÁLISIS DE EFICIENCIA DE ALGORITMOS

            
            \Large
            \vspace{1cm}
            \textbf{Integrantes:}  \\ 
             % Array con los nombres de los integrantes y el correo
             \begin{center}
                \begin{tabular}{c c }
                    \textbf{Miguel Ángel De la Vega Rodríguez} & Miguevrod@correo.ugr.es \\
                    \textbf{Alberto De la Vera Sánchez} & Joaquin724@correo.ugr.es \\
                    \textbf{Joaquín Avilés De la Fuente} & Adelaveras01@correo.ugr.es \\
                    \textbf{manu } & manu@correo.ugr.es \\
                    \textbf{Pablo} & pablo@correo.ugr.es
                \end{tabular}
             \end{center}
            \vspace{0.8cm}
            
            
            \large
             \vspace{1cm}
            Facultad de Ciencias UGR\\
            Escuela Técnica Ingeniería Informática UGR\\
            Granada\\
            #2 
            
        \end{center}
    \end{titlepage}
}



\usepackage{assets/formulas}
\usepackage{float}
\hbadness=10000 % Suppress Underfull \hbox warnings

%========================================|Indice|===============================================%

\begin{document}
\portada{Algorítmica}{2023-2024}{Miguel Ángel De la Vega Rodríguez}{https://github.com/Miguevrgo/}{github.png}
\tableofcontents % Índice
\newpage %Salto de pagina tras el Indice


%======================================|Documento|==============================================%
\chapter{Autores}
\begin{itemize}
      \item \textbf{Miguel Ángel De la Vega Rodríguez:} 20\%
            \begin{itemize}
                  \item Algoritmos del Viajante
                  \item Redacción memoria sección Viajante
            \end{itemize}
      \item \textbf{Joaquín Avilés De la Fuente:} 20\%
            \begin{itemize}
                  \item Programacion SumaMax (DyV)
                  \item Estudio eficiencia teórica algoritmos específicos y DyV
                  \item Tests de eficiencia
            \end{itemize}
      \item \textbf{Alberto De la Vera Sánchez: } 20\%
            \begin{itemize}
                  \item Redacción \LaTeX
                  \item Estudio de umbrales teóricos y empíricos de DyV
                  \item Graficas y ajustes
            \end{itemize}
      \item \textbf{Manuel Gomez Rubio} 20\%
            \begin{itemize}
                  \item Programacion SumaMax (Kadane)
                  \item Programacion Losetas
            \end{itemize}
      \item \textbf{Pablo Linari Pérez:} 20\%
            \begin{itemize}
                  \item Programacion SumaMax (Kadane)
                  \item Programacion Losetas
            \end{itemize}
\end{itemize}

\chapter{Equipo de trabajo}

\begin{itemize}
      \item \textbf{Miguel Ángel De la Vega Rodríguez:} (Ordenador donde se ha realizado el computo)
            \begin{itemize}
                  \item AMD Ryzen 7 2700X 8-Core
                  \item 16 GB RAM DDR4 3200 MHz
                  \item NVIDIA GeForce GTX 1660 Ti
                  \item 1 TB SSD NvMe
                  \item Debian 12 Bookworm
                  \item Compilador GCC 12.2.0
            \end{itemize}
\end{itemize}
% Viajante es el apartado 6, incluir capitulos antes
\chapter{Viajante} % Migue
En esta sección se analiza todo aquello referente al cuarto problema propuesto,
el problema del viajante. Siguiendo las directrices indicadas, se han estudiado
y diseñado diferentes algoritmos "Greedy" que aproximan el problema, con el objetivo
final de comparar cual nos proporciona mejores resultados en cuanto a términos
de eficiencia y precisión. Para determinar esto último, también cabe destacar
detalles de menor importancia como la complejidad de los algoritmos, la estabilidad
frente a distintas entradas o la posibilidad de mejora de los mismos mediante la
elección de distintos puntos de partida o con mejoras posteriores como pueden ser 
aquellas que proporcionan algoritmos como $\lambda$-opt o genéticos.
\\ \\
De aqui en adelante, describiremos por secciones los algoritmos implementados y 
al final proporcionaremos una sección comparativa sobre la cual nos basaremos
para determinar conclusiones, en particular, elegiremos el algoritmo cuya
solución consideremos más conveniente. En lo que sigue se muestra únicamente
el resultado obtenido por el algoritmo sin mejoras posteriores, este tipo de mejoras
suponen una pequeña desvirtuación del objetivo de encontrar el mejor algoritmo y 
es por ello, que su uso se limita a la conclusión final.

\section{Algoritmo \textit{Nearest Neighbour}}
Tal y como el nombre indica, el primer algoritmo implementado no es nada más, ni 
nada menos que el primer algoritmo que probablemente se le puede ocurrir a cualquier
persona que se enfrente a este problema. La idea es sencilla e intuitiva, nos 
dan un conjunto de puntos y queremos encontrar el camino más corto que los recorra,
para ello, \textbf{elegimos} un punto inicial y a partir de ese punto, en un proceso 
iterativo, elegimos en cada paso el siguiente punto más cercano al actual, o lo que 
es lo mismo, el punto para el cual la distancia al punto actual es la menor (No confundir
con el punto para el cual la distancia total es la menor, ya que, aunque parecido, no es 
lo mismo). Este proceso se repite hasta que todos los puntos han sido visitados.
\\ \\
A continuación se proporciona la implementación propuesta del algoritmo en cuestión:
\begin{lstlisting}[language=C++]
vector<Point> nearestNeighborTSP(const vector<Point>& points) {
  vector<Point> path;
  vector<bool> visited(points.size(), false);
  path.reserve(points.size());
  path.emplace_back(points[0]);
  visited[0] = true;

  for (int i = 0; i < points.size() - 1; ++i) {
    double minDistance = numeric_limits<double>::max();
    int nearestNeighbor = -1;
    for (int j = 0; j < points.size(); ++j) {
      if (!visited[j]) {
        double distance = path[i].distanceTo(points[j]);
        if (distance < minDistance) {
          minDistance = distance;
          nearestNeighbor = j;
        }
      }
    }
    path.emplace_back(points[nearestNeighbor]);
    visited[nearestNeighbor] = true;
  }

  return path;
}
\end{lstlisting}
Como se puede apreciar, se ha hecho uso de un vector de puntos visitados (Programación 
dinámica) para evitar visitar un punto más de una vez, esto no supone una complejidad
notable, sin embargo, si que supone una gran mejora en términos de eficiencia.

\section{Algoritmo \textit{Ordenación}}
Como el nombre indica, este algoritmo consiste en ordenar los puntos de partida de
acuerdo a un criterio especifico (en nuestro caso los hemos ordenado por la coordenada
x de menor a mayor, teniendo en cuenta también para puntos cercanos en x, la coordenada y).
Una vez ordenados los puntos, simplemente recorremos el vector de puntos en el orden
en el que se encuentren, de esta forma, el camino total trata de reducir cruces y distancias
con la idea intuitiva de que si dos puntos están cerca en el plano, probablemente, la 
distancia que se recorra al pasar por ellos, sea mínima. Como veremos en el siguiente 
algoritmo y en la conclusión, esto en la práctica no es del todo cierto. Primero veamos 
la implementación del algoritmo:
\begin{lstlisting}[language=C++]
vector<Point> orderedTSP(const vector<Point>& points) {
      vector<Point> tour;
      tour.reserve(points.size());
      copy(points.begin(), points.end(), back_inserter(tour));
      sort(tour.begin(), tour.end());
        
      return tour;
}      
\end{lstlisting}
Como se puede apreciar, la implementación es muy sencilla, simplemente se copian los puntos
de partida a un vector auxiliar, se ordenan y se devuelve el vector ordenado, sin embargo
cuando vemos el resultado de la ejecución, nos damos cuenta de que aunque los puntos
esten muy cerca respecto a la coordenada x, los saltos que se dan en la coordenada y
pueden ser muy grandes, lo que hace que el camino total sea mucho mayor de lo esperado,
además para cerrar el camino, la distancia es la máxima posible en cuanto a x. Para mejorar
esto, podemos observar que si nuestro problema son los saltos en la coordenada y 
y la distancia de cierre, podemos intentar minimizarlos, de donde surge el siguiente
algoritmo.

\section{Algoritmo \textit{Circular}}

\end{document}