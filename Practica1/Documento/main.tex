%_________________________________________Heading_______________________________________________%

\documentclass[11pt]{article}
% #1-Asignatura
% #2-Curso
% #3-Nombre
% #4-Link
% #5-Foto

\newcommand{\portada}[5]{
    \begin{titlepage}
        \begin{center}
            \vspace*{0.5cm}
            
            % Titulo con #1 lo mas grande posible
            {\Huge \textbf{#1}}

            
            \vspace{0.5cm}
            \LARGE
            Curso #2 
            
            \vspace{1cm}
            
            \Huge{\textbf{Grupo Viterbi}}

            \vspace{1cm}
            \includegraphics[width=0.6\textwidth]{assets/Img/UGR-Logo.png}
            
            \vspace{0.5cm}

            \huge
            PRÁCTICA 1-ANÁLISIS DE EFICIENCIA DE ALGORITMOS

            
            \Large
            \vspace{1cm}
            \textbf{Integrantes:}  \\ 
             % Array con los nombres de los integrantes y el correo
             \begin{center}
                \begin{tabular}{c c }
                    \textbf{Miguel Ángel De la Vega Rodríguez} & Miguevrod@correo.ugr.es \\
                    \textbf{Alberto De la Vera Sánchez} & Joaquin724@correo.ugr.es \\
                    \textbf{Joaquín Avilés De la Fuente} & Adelaveras01@correo.ugr.es \\
                    \textbf{manu } & manu@correo.ugr.es \\
                    \textbf{Pablo} & pablo@correo.ugr.es
                \end{tabular}
             \end{center}
            \vspace{0.8cm}
            
            
            \large
             \vspace{1cm}
            Facultad de Ciencias UGR\\
            Escuela Técnica Ingeniería Informática UGR\\
            Granada\\
            #2 
            
        \end{center}
    \end{titlepage}
}



\usepackage{assets/formulas}
\newcommand{\negrita}[1]{\textbf{#1}}
\hbadness=10000 % Suppress Underfull \hbox warnings

%_________________________________________Indice:_______________________________________________%
\begin{document}                                                
\portada{Algorítmica}{2023-2024}{Miguel Ángel De la Vega Rodríguez}{https://github.com/Miguevrgo/}{github.png}

\tableofcontents % Índice

\newpage %Salto de pagina tras el Indice

%________________________________________Documento:_____________________________________________%
\section{Participación}
\begin{itemize}
    \item \textbf{Miguel Ángel De la Vega Rodríguez:} 20\%
    \begin{itemize}
        \item Plantilla y estructura del documento \LaTeX
        \item Cómputo de la eficiencia de los algoritmos (Resultados y Ajuste)
    \end{itemize}
    \item \textbf{Joaquín Avilés De la Fuente:} 20\%
    \begin{itemize}
        \item Descripción del Objetivo de la pŕactica
        \item Diseño del estudio
    \end{itemize}
    \item \textbf{Alberto De la Vera Sánchez: } 20\%
    \item \textbf{Manuel Gomez Rubio} 20\%
    \item \textbf{Pablo Linari Perez:} 20\%
\end{itemize}

\section{Equipo de trabajo}

\begin{itemize}
    \item \textbf{Miguel Ángel De la Vega Rodríguez:} (Ordenador donde se ha realizado el computo)
        \begin{itemize}
            \item AMD Ryzen 7 2700X 8-Core
            \item 16 GB RAM DDR4 3200 MHz
            \item NVIDIA GeForce GTX 1660 Ti 
            \item 1 TB SSD NvMe 
        \end{itemize}
\end{itemize}

\section{Objetivos}
    En esta práctica, se han implementado los siguientes algoritmos de ordenación: \negrita{quicksort, mergesort, heapsort, inserción, burbuja,}
    y \negrita{selección}. Además, se han implementado los algoritmos de \negrita{Floyd}, que calcula el costo del camino mínimo entre cada par de nodos 
    de un grafo dirigido, de \negrita{Fibonacci}, que calcula los números de la sucesión de Fibonacci , y de \negrita{Hanoi}, que resuelve el famoso 
    problema de las torres de Hanoi. \\ \\
    En primer lugar, aunque tenemos la eficiencia teórica de estos algoritmos, se realizarán los calculos necesarios para demostrar
    como se obtiene dicha eficiencia utilizando los distintos métodos estudiados en teoría. \\
    En segundo lugar, se pasará al estudio empírimo de los algoritmos de ordenación de vectores para distintos tipos de datos, es decir, 
    para datos tipo \negrita{int}, \negrita{float}, \negrita{double} y \negrita{string}. Posteriormente, se creará las gráficas para
    cada algoritmo en las que visualizaremos el tiempo de ejecución en función del tamaño del vector y del tipo de dato. Finalmente 
    para esta parte, se hara un calculo de eficiencia híbrido que se basa en ajustar la gráfica obtenida a la función de su eficiencia
    teórica por mínimos cuadrados, obteniendo por tanto los literales de dicha función que ajustan la gráfica.\\
    En tercer lugar, se hará el estudio de los otros tres algoritmos de forma similar, es decir, se estudiará la eficiencia
    de estos de modo empírica, cuyo estudio se mostrará en las gráficas, y se calculará la eficiencia híbrida de estos, a partir
    de la eficiencia teórica.\\ 

\section{Diseño del estudio}
% Esta variable indica el algoritmo escogido para el estudio en varios ordenadores
    \newcommand{\mivar}{ordenación de vectores quicksort}
    
\section{Algoritmos}
\section{Conclusiones}

\end{document}
