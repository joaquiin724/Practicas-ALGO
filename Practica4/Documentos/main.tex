\documentclass[11pt,openany]{book}
% #1-Asignatura
% #2-Curso
% #3-Nombre
% #4-Link
% #5-Foto

\newcommand{\portada}[5]{
    \begin{titlepage}
        \begin{center}
            \vspace*{0.5cm}
            
            % Titulo con #1 lo mas grande posible
            {\Huge \textbf{#1}}

            
            \vspace{0.5cm}
            \LARGE
            Curso #2 
            
            \vspace{1cm}
            
            \Huge{\textbf{Grupo Viterbi}}

            \vspace{1cm}
            \includegraphics[width=0.6\textwidth]{assets/Img/UGR-Logo.png}
            
            \vspace{0.5cm}

            \huge
            PRÁCTICA 1-ANÁLISIS DE EFICIENCIA DE ALGORITMOS

            
            \Large
            \vspace{1cm}
            \textbf{Integrantes:}  \\ 
             % Array con los nombres de los integrantes y el correo
             \begin{center}
                \begin{tabular}{c c }
                    \textbf{Miguel Ángel De la Vega Rodríguez} & Miguevrod@correo.ugr.es \\
                    \textbf{Alberto De la Vera Sánchez} & Joaquin724@correo.ugr.es \\
                    \textbf{Joaquín Avilés De la Fuente} & Adelaveras01@correo.ugr.es \\
                    \textbf{manu } & manu@correo.ugr.es \\
                    \textbf{Pablo} & pablo@correo.ugr.es
                \end{tabular}
             \end{center}
            \vspace{0.8cm}
            
            
            \large
             \vspace{1cm}
            Facultad de Ciencias UGR\\
            Escuela Técnica Ingeniería Informática UGR\\
            Granada\\
            #2 
            
        \end{center}
    \end{titlepage}
}



\usepackage{assets/formulas}
\usepackage[mathscr]{euscript}
\usepackage{bm}
\usepackage{float}
\hbadness=10000 % Suppress Underfull \hbox warnings

%========================================|Indice|===============================================%

\begin{document}
\portada{Algorítmica}{2023-2024}{Miguel Ángel De la Vega Rodríguez}{https://github.com/Miguevrgo/}{github.png}
\tableofcontents % Índice
\newpage %Salto de pagina tras el Indice


%======================================|Documento|==============================================%
\chapter{Autores}
\begin{itemize}
      \item \textbf{Miguel Ángel De la Vega Rodríguez:} 20\%
            \begin{itemize}
                  \item Algoritmos Greedy del Viajante
                  \item Redacción memoria sección Viajante
            \end{itemize}
      \item \textbf{Joaquín Avilés De la Fuente:} 20\%
            \begin{itemize}
                  \item Programacion Dijkstra
                  \item Programación creación de grafos (casos de prueba)
                  \item Redacción memoria sección Dijkstra
                  \item Creación de gráficas y estudio de eficiencia Dijkstra
            \end{itemize}
      \item \textbf{Alberto De la Vera Sánchez: } 20\%
            \begin{itemize}
                  \item Programación hijo predilecto
                  \item Redacción hijo predilecto
                  \item Conclusión
            \end{itemize}
      \item \textbf{Manuel Gomez Rubio} 20\%
            \begin{itemize}
                  \item Redacción y demostración del algorimo greedy
                  \item Redacción memoria del algoritmo Dijkstra
                  \item Programacion Dijkstra
                  \item Creación de gráficas y estudio de eficiencia Dijkstra
            \end{itemize}
      \item \textbf{Pablo Linari Pérez:} 20\%
            \begin{itemize}
                  \item Programacion Aulas
                  \item Redacción Aulas
                  \item Redacción objetivos
            \end{itemize}
\end{itemize}

\chapter{Equipo de trabajo}

\begin{itemize}
      \item \textbf{Miguel Ángel De la Vega Rodríguez:} (Ordenador donde se ha realizado el computo)
            \begin{itemize}
                  \item AMD Ryzen 7 2700X 8-Core
                  \item 16 GB RAM DDR4 3200 MHz
                  \item NVIDIA GeForce GTX 1660 Ti
                  \item 1 TB SSD NvMe
                  \item Debian 12 Bookworm
                  \item Compilador GCC 12.2.0
            \end{itemize}
\end{itemize}

\chapter{Objetivos}
    El objetivo de esta práctica es resolver el problema del viajante de comercio el cual viene descrito por el siguiente enunciado :\
    Tenemos un conjunto de n ciudades (puntos en un plano), cada una definida por las coordenadas en el mapa $(x_i , y_i )$, con $i = 1, . . . , n$. La distancia entre dos ciudades viene dada por la
    distancia euclídea entre sus coordenadas.
    El problema de viajante de comercio consiste en encontrar el orden en el que un viajante,
    partiendo de la ciudad de origen (por ejemplo $(x_1 , y_1 )$) pase por todas y cada una de las ciudades
    una única vez, para volver a la ciudad de partida, formando un ciclo. \ 

    Para ello usaremos dos diseños distintos de  algoritmos dedicados a la exploraciónde grafos , backtracking y brannch and bound 
    con el objetivo de ver las diferencias entre la eficiencia de estos dos algoritmos. Además se probarán distintas funciones de cotsa 
    para estudiar que influencia tienen en cada caso las distintas funciones.Por último se realizará un estudio tanto teórico como 
    empírico de la eficiencia de los algoritmos implementados.
\chapter{Backtracking}   
\section{Diseño}   
En esta sección analizaremos el algorimo de exploración de grafos llamado Backtracking, que consiste en recorrer todos los caminos en profundidad obteniendo todas
las posibles soluciones al problema. \\
Para seleccionar la solución más óptima del problema debemos ir comparando con la anterior posible solución  cada vez que obtenemos una nueva solución al problema, quedándonos con la que más nos
convenga, en este caso, la que nos de un camino de menor disatancia.
\\ \\
Hay varias formas de hacerlo:
\begin{itemize}
      \item La primera forma consiste en hacerlo mediante fuerza bruta, es decir, usar la definición al uso de Backtracking recorriendo todos los caminos sin preocuparnos
      si estos nos permiten alcanzar una solución mejor que la obtenida anteriormente.
      \item La otra forma es implementar una función de cota, esto nos permite decidir si seguimos explorando el camino seleccionado porque nos puede dar un resultado mejor
      o si es mejor abandonarlo, ya que no se obtendrá una mejora.
      Esta forma de realizarlo será, como es lógico, más eficiente.
\end{itemize}

Para la implementación del algoritmo, optamos por la segunda forma, usando diferentes funciones de cota que nos permitan aproximar si el camino seleccionado es bueno.A continuación
veremos como se ha elegido cada cota .
\subsection{Cota global}
\subsection{Cota local}
\textbf{Primera función de cota:} \\
    La primera función de cota implementada consiste en seleccionar, de todos los posibles puntos  que no hayan sido recorridos con aterioridad, el que esté a menor distancia, y tomarlo como si el resto tuvieran esa misma distancia,
    es decir, si hemos seleccionado ya 3 puntos y quedan 2 mas por recorrer, la función de cota interpretaría que la distancia total que nos quedaría por recorrer será dos veces, ya que sólo quedan dos puntos por seleccionar, la menor de esos dos puntos al último seleccionado.
    De esta forma si la solución que se propone mediante esta función de cota no es mejor que la que tengamos seleccionada en ese momento como la mejor, entonces se descartará el camino y se comenzará a explorar otro.
\subsection{Algoritmo Backtracking}
\section{Justificación}
\section{Eficiencia teórica y empírica}
\subsection{Eficiencia teórica}




\chapter{Branch and bound}
Al igual que en el apartado anterior resolveremos el problema del viajante, solo que en vez de solucionar el problema mediante el bakctraking 
lo resolveremos mediante Branch and bound. A continuación se mostrarán las funciones de cota al igual que el algoritmo principal que hemos usado:

\section{Diseño}
\subsection{Cota global}

Primero empezaremos mostrando la función de cota global:
\begin{lstlisting}
vector<int> nearest_neighborTSP(const vector<vector<double>>& distancias, int inicial) {
    int num_puntos = distancias.size();
    vector<int> camino;
    vector<bool> visitados(num_puntos, false);
    camino.reserve(num_puntos);
    camino.push_back(inicial);
    visitados[inicial] = true;

    for (int i = 0; i < num_puntos - 1; ++i) {
      int actual = camino.back();
      int siguiente = -1;
      double min_distancia = numeric_limits<double>::max();

      for (int j = 0; j < num_puntos; ++j) {
          if (!visitados[j] && distancias[actual][j] < min_distancia) {
              min_distancia = distancias[actual][j];
              siguiente = j;
          }
      }

      camino.push_back(siguiente);
      visitados[siguiente] = true;
    }

    camino.push_back(inicial);
    return camino;
}

\end{lstlisting} 
Esta función es obtenida de la práctica anterior de algoritmos Greedy. La idea es sencilla, nos dan un conjunto de 
puntos y queremos encontrar el camino más corto que los recorra, para ello, elegimos un punto inicial y a partir de 
ese punto, en un proceso iterativo, elegimos en cada paso el siguiente punto más cercano al actual, o lo que es lo mismo, 
el punto para el cual la distancia al punto actual es la menor. Este proceso se repite hasta que 
todos los puntos han sido visitados.
Este algoritmo Greey junto con la siguiente función, que calcula la distancia total de un camino dado a partir de su matriz
de adyacencia, nos permite obtener una cota superior inicial.
\\ 
\begin{lstlisting}
double calcularDistanciaTotal(const vector<vector<double>>& distancias, const vector<int>& camino) {
    double total = 0.0;
    for (int i = 0; i < camino.size() - 1; ++i) {
        total += distancias[camino[i]][camino[i + 1]];
    }
    total += distancias[camino.back()][camino.front()];
    return total;
}
      
\end{lstlisting}
\subsection{Cota local}

Además de esta cota global, hemos implementado do funciones de cota local distintas:
\\ 
\begin{lstlisting}
double cota_inferior_1(const std::vector<std::vector<double>>& matriz, const std::vector<int>& indices_a_ignorar = {}) {
    double suma_minimos = 0.0;

    for (int i = 0; i < matriz.size(); ++i) {
        
        if (std::find(indices_a_ignorar.begin(), indices_a_ignorar.end(), i) != indices_a_ignorar.end()) {
            continue;
        }

        double minimo_fila = std::numeric_limits<double>::max();
        for (int j = 0; j < matriz[i].size(); ++j) {
            
            if (std::find(indices_a_ignorar.begin(), indices_a_ignorar.end(), j) != indices_a_ignorar.end()) {
                continue;
            }
            if (matriz[i][j] > 0 && matriz[i][j] < minimo_fila) {
                minimo_fila = matriz[i][j];
            }
        }

        if (minimo_fila < std::numeric_limits<double>::max()) {
            suma_minimos += minimo_fila;
        }
    }

    return suma_minimos;
}    

double cota_inferior_2(const std::vector<std::vector<double>>& matriz, const std::vector<int>& indices_a_ignorar = {}) {
    double min_global = std::numeric_limits<double>::max();
    int filas_contadas = 0;

    for (int i = 0; i < matriz.size(); ++i) {
        
        if (std::find(indices_a_ignorar.begin(), indices_a_ignorar.end(), i) != indices_a_ignorar.end()) {
            continue;
        }

        double minimo_fila = std::numeric_limits<double>::max();
        for (int j = 0; j < matriz[i].size(); ++j) {
            
            if (std::find(indices_a_ignorar.begin(), indices_a_ignorar.end(), j) != indices_a_ignorar.end()) {
                continue;
            }
            if (matriz[i][j] > 0 && matriz[i][j] < minimo_fila) {
                minimo_fila = matriz[i][j];
            }
        }

        if (minimo_fila < std::numeric_limits<double>::max()) {
            if (minimo_fila < min_global) {
                min_global = minimo_fila;
            }
            ++filas_contadas;
        }
    }

    return  (min_global * filas_contadas);
}
            
\end{lstlisting}
La primera función de cota local consiste en calcular el mínimo de una matriz de distancias, excluyendo obviamente los valores
de 0, pasandose como parámetros los índices a ignorar.Destacar que además de obviar dichas filas se obviarán sus respectivas 
columnas indicadas en el vector de enteros, ya que el objetivo es eliminar las conexiones de ciertos nodos, por lo que 
eliminamos sus filas y columnas respectivas.
\\ \\
Por otro lado, la segunda función de cota, que es la que vamos a utilizar en nuestro algoritmo Branch and Bound, tiene un funcionamiento 
distinto. Consiste principalmente en encontrar la mínima distancia de una matriz de adyacencia y depués multiplicar dicho valor por
las filas restantes, suponiendo que ese camino será el mínimo para todos, de ahí que se multiplique por el número de nodos restantes.
Ignorando, igual que con la cota anterior, los 0 y las filas y las columnas que se pasan como parámetros a ignorar. Función para calcular 
cota inferior calculando el mínimo de una matriz de distancias, excluyendo el 0 y las filas. Es claro que este segundo método nos proporciona 
una cota más restrictiva por lo que nos interesa más.
\\ \\
Más adelante en el apartado de eficiencia seguiremos comparando dichas cotas.

\subsection{Algoritmo Branch and Bound}
Primero de todo, el struct con el que trabajaremos para almacenar la información sobre la cota mínima, el camino y la distancia recorrida será:

\begin{lstlisting}
      struct Nodo {
            vector<int> path;
            double distancia_recorrida; 
            double cota_inferior; 
        
            Nodo(vector<int>& inicial, double dist_rec, double cota_inf) {
                path = inicial;
                distancia_recorrida = dist_rec;
                cota_inferior = cota_inf;
            }
      };
\end{lstlisting}

Ahora, el algoritmo Brach and Bound que hemos implementado ha sido:

\begin{lstlisting}
vector<int> branch_and_bound_greedy(vector<int>& points, vector< vector<double>> &distancias, int inicial){
    priority_queue<Nodo, vector<Nodo>, Comparador> no_visitados;
    vector<int> mejor_camino = {inicial};
    Nodo actual( mejor_camino, 0, cota_inferior(distancias));
    no_visitados.push(actual);

    double costo_minimo = calcularDistanciaTotal(distancias, nearest_neighborTSP(distancias, inicial));
    mejor_camino.clear();

    while (!no_visitados.empty()) {
        actual = no_visitados.top();
        no_visitados.pop();

        if (actual.path.size() == points.size()-1) {
            vector<int> faltantes= numeros_faltantes(actual.path, points.size()-1);
            actual.distancia_recorrida += distancias[actual.path.back()][faltantes[0]];
            actual.distancia_recorrida += distancias[faltantes[0]][inicial];
            actual.path.push_back(faltantes[0]);
            actual.path.push_back(inicial);
            if (actual.distancia_recorrida <= costo_minimo) {
                costo_minimo = actual.distancia_recorrida;
                mejor_camino = actual.path;
            }
        } 
        else { 
            if (actual.cota_inferior <= costo_minimo ){
                vector<int> faltantes = numeros_faltantes(actual.path, points.size()-1);
                for (int i = 0; i < faltantes.size(); ++i) {
                    Nodo nuevo= actual;
                    nuevo.path.push_back(faltantes[i]);
                    nuevo.distancia_recorrida += distancias[actual.path.back()][faltantes[i]];
                    nuevo.cota_inferior = nuevo.distancia_recorrida + cota_inferior(distancias, nuevo.path);
                    no_visitados.push(nuevo);
                }
            }
        }
    }

    return mejor_camino;

}
\end{lstlisting}
El funcionamiento de este método se basa en empezar creando un nodo con el punto inicial y agregarlo a la cola de prioridad (cuyo comparador
consiste en comparar la distancia recorrida por un nodo a y un nodo b y ver qué distancia es menor). Mientras la cola no esté vacía; primero,
se extrae el nodo con menor costo actual (a partir de la cota inferior y haciendo uso del comparador definido anteriormente); si el camino actual 
tiene todos los puntos menos unos, sabemos de forma determinada cual queda mediante la función faltantes de antes(dicha función encuentra los números
faltantes desde 0 hasta n en un vector de enteros), por lo que se calcula la distancia a dicho punto y al inicial y se agrega a la solución si es 
mejor que la actual; si no, se generan los nodos hijos con los puntos faltantes y se agregan a la cola de prioridad.
\\ \\
A continuación se mostrará la notación mostrada en teoría, aunque sigamos un esquema similar, hay ciertos ámbitos que pueden ser suprimidos.
\\\\
\textbf{Notación:}
\begin{itemize}
      \item \textbf{Solución parcial:} Vector path del Nodo actual.
      \item \textbf{Función poda:} Momento en que la cota inferior supera al costo minimo,siendo este la cota global.
      \item \textbf{Restricciones explícitas:}
      \item \textbf{Restricciones implícitas:}
      \item \textbf{Árbol de estado:} El espacio solución con el que se trabaja es el Nodo actual, sin embargo, el 
      vector mejor camino es el árbol de estado al encontrar la solución.
      \item \textbf{Estado del problema:} Cada uno de los nodos del árbol
      \item \textbf{Estado solución:} Nodo actual
      \item \textbf{Estado respuesta:}Vector mejor camino
      \item \textbf{Nodo vivo:}
      \item \textbf{Nodo muerto:}
      \item \textbf{e-nodo:}Nodo actual
\end{itemize}


\newpage
\section{Justificación}

\newpage

\section{Eficiencia teórica y empírica}
\subsection{Eficiencia teórica}

En primer lugar, como ya calculamos en la práctica anterior, tenemos que la eficiencai de la cota global es de $O(n^2)$




\end{document}