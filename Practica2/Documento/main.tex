\documentclass[11pt,openany]{book}
% #1-Asignatura
% #2-Curso
% #3-Nombre
% #4-Link
% #5-Foto

\newcommand{\portada}[5]{
    \begin{titlepage}
        \begin{center}
            \vspace*{0.5cm}
            
            % Titulo con #1 lo mas grande posible
            {\Huge \textbf{#1}}

            
            \vspace{0.5cm}
            \LARGE
            Curso #2 
            
            \vspace{1cm}
            
            \Huge{\textbf{Grupo Viterbi}}

            \vspace{1cm}
            \includegraphics[width=0.6\textwidth]{assets/Img/UGR-Logo.png}
            
            \vspace{0.5cm}

            \huge
            PRÁCTICA 1-ANÁLISIS DE EFICIENCIA DE ALGORITMOS

            
            \Large
            \vspace{1cm}
            \textbf{Integrantes:}  \\ 
             % Array con los nombres de los integrantes y el correo
             \begin{center}
                \begin{tabular}{c c }
                    \textbf{Miguel Ángel De la Vega Rodríguez} & Miguevrod@correo.ugr.es \\
                    \textbf{Alberto De la Vera Sánchez} & Joaquin724@correo.ugr.es \\
                    \textbf{Joaquín Avilés De la Fuente} & Adelaveras01@correo.ugr.es \\
                    \textbf{manu } & manu@correo.ugr.es \\
                    \textbf{Pablo} & pablo@correo.ugr.es
                \end{tabular}
             \end{center}
            \vspace{0.8cm}
            
            
            \large
             \vspace{1cm}
            Facultad de Ciencias UGR\\
            Escuela Técnica Ingeniería Informática UGR\\
            Granada\\
            #2 
            
        \end{center}
    \end{titlepage}
}



\usepackage{assets/formulas}
\hbadness=10000 % Suppress Underfull \hbox warnings

%========================================|Indice|===============================================%

\begin{document}
\portada{Algorítmica}{2023-2024}{Miguel Ángel De la Vega Rodríguez}{https://github.com/Miguevrgo/}{github.png}
\tableofcontents % Índice
\newpage %Salto de pagina tras el Indice


%======================================|Documento|==============================================%
\chapter{Autores}
\begin{itemize}
    \item \textbf{Miguel Ángel De la Vega Rodríguez:} 20\%
          \begin{itemize}
              \item Plantilla y estructura del documento \LaTeX
          \end{itemize}
    \item \textbf{Joaquín Avilés De la Fuente:} 20\%
          \begin{itemize}
            \item Tarea
          \end{itemize}
    \item \textbf{Alberto De la Vera Sánchez: } 20\%
          \begin{itemize}
            \item Tarea
          \end{itemize}
    \item \textbf{Manuel Gomez Rubio} 20\%
          \begin{itemize}
            \item Tarea
          \end{itemize}
    \item \textbf{Pablo Linari Pérez:} 20\%
          \begin{itemize}
            \item Tarea
          \end{itemize}
\end{itemize}

\chapter{Objetivos}
En esta práctica, se pretende resolver problemas de forma eficiente aplicando la técnica de
Divide y Vencerás. Para ello, se han planteado varios problemas cuya solución es conocida
(excepto para el problema del viajante), y se han implementado algoritmos que los resuelven
mediante el método convencional y mediante la técnica de Divide y Vencerás. Posteriormente, se ha buscado
un umbral en el cual ambos tengan el mismo tiempo de ejecución, finalmente, se ha buscado el 
umbral óptimo para cada problema. 
\chapter{Definicion Problema}
\chapter{Algoritmo Especifico}
\chapter{Algoritmo Divide y Vencerás}
\chapter{Conclusiones}
\end{document}